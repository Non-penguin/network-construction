\documentclass[a4paper,11pt]{report}
\usepackage[utf8]{inputenc}
\usepackage{geometry}      % 余白調整用
\usepackage{booktabs}      % きれいな表用
\usepackage{amsmath}       % 数式用
\usepackage{fancyhdr}      % ヘッダー・フッター用

% 余白の設定
\geometry{left=25mm, right=25mm, top=30mm, bottom=30mm}

% タイトル設定
\title{\textbf{\Huge IPv4を速攻で攻略しよう}\\ \large 脳死で覚えるサブネット設計}
\author{Nozomu Wada}
\date{\today}

\begin{document}

\maketitle
\tableofcontents
\thispagestyle{empty}
\clearpage

% 第1部:はじめに
\chapter{はじめに}

\section{目的}
とりあえずネットワークに関するテストを乗り切りたい人、忙しい人に向けた「実践的」な解説書です。
深い理論は一旦置いておき、\textbf{「どう計算すれば答えが出るか」}にフォーカスします。

\section{前提知識}
このレポートを読むにあたり、以下の知識があったほうがいいかもです。
\begin{itemize}
    \item 二進数の計算ができること
    \item サブネットマスクの概念をなんとなく知っていること
    \item CIDR表記(/24など)をなんとなく知っていること
\end{itemize}

\section{今回の例題}
スケール \textbf{10.10.0.0/16} のネットワークを、以下の要件に合わせて分割します。

\begin{table}[h]
    \centering
    \caption{ネットワーク要件}
    \label{tab:requirements}
    \begin{tabular}{lcr}
        \toprule
        \textbf{拠点名} & \textbf{必要ホスト数} & \textbf{備考} \\
        \midrule
        Factory & 3000 &  \\
        Office  & 1000 &  \\
        Lab     & 120  &  \\
        Link    & 2    & 二拠点間接続 \\
        \bottomrule
    \end{tabular}
\end{table}

\clearpage
