\documentclass[a4paper,11pt]{report}
\usepackage[utf8]{inputenc}
\usepackage{geometry}
\usepackage{booktabs}
\usepackage{amsmath}
\usepackage{fancyhdr}
\usepackage{luatexja}

% 余白の設定
\geometry{left=25mm, right=25mm, top=30mm, bottom=30mm}

% タイトル設定
\title{\textbf{\Huge IPv4を速攻で攻略しよう}\\ \large 脳死で覚えるサブネット設計}
\author{Nozomu Wada}
\date{\today}

\begin{document}

\maketitle
\tableofcontents
\thispagestyle{empty}
\clearpage

% 第1部:はじめに
\chapter{はじめに}

%1.0
\section{目的}
とりあえずネットワークに関するテストを乗り切りたい人、忙しい人に向けた「実践的」な解説書です。
深い理論は一旦置いておき、\textbf{「どう計算すれば答えが出るか」}にフォーカスします。

%1.1
\section{前提知識}
このレポートを読むにあたり、以下の知識があったほうがいいかもです。
\begin{itemize}
    \item 二進数の計算ができること
    \item サブネットマスクの概念をなんとなく知っていること
    \item CIDR表記(/24など)をなんとなく知っていること
\end{itemize}

\subsection{!!注意!!}
実践的ではあるが、とりあえずテストをパスしたい、なんとなく理解したい人むけ

%1.2
\section{今回の例題}
スケール \textbf{10.10.0.0/16} のネットワークを、以下の要件に合わせて分割します。

\begin{table}[h]
    \centering
    \caption{ネットワーク要件}
    \label{tab:requirements}
    \begin{tabular}{lcr}
        \toprule
        \textbf{拠点名} & \textbf{必要ホスト数} & \textbf{備考} \\
        \midrule
        Factory & 3000 &  \\
        Office  & 1000 &  \\
        Lab     & 120  &  \\
        Link    & 2    & 二拠点間接続 \\
        \bottomrule
    \end{tabular}
\end{table}

\clearpage

% 第2部:ホスト数からサブネットマスクを決定する
\chapter{ホスト数からサブネットマスクを決定する}

\section{Factoryの設計}
\subsection{Step1:サブネットマスクを求める}
Factoryは3000台のホストが必要。つまり、サブネットマスクは\[2^n-2 \ge 3000\]を満たす最小のnを求める。
\section{Officeの設計}
\subsection{Step1:サブネットマスクを求める}
Officeは1000台のホストが必要。サブネットマスクは $2^n-2 \geq 1000$ を満たす最小のnを求める。
\begin{align*}
    2^{10} - 2 &= 1022 \\
    2^{9} - 2 &= 510
\end{align*}
よって、n=10が必要。サブネットマスクは32-10=22、つまり/22となる。

\subsection{Step2:ネットワークアドレスを求める}
Factoryの次、10.0.16.0/22 から始まる。

\subsubsection{暗算テクニック}
/22は第三オクテットで4ずつ増加。10.0.16.0, 10.0.20.0, ...
Officeの次のネットワークアドレスは10.0.20.0。

\section{Labの設計}
\subsection{Step1:サブネットマスクを求める}
Labは120台のホストが必要。サブネットマスクは $2^n-2 \geq 120$ を満たす最小のnを求める。
\begin{align*}
    2^{7} - 2 &= 126 \\
    2^{6} - 2 &= 62
\end{align*}
よって、n=7が必要。サブネットマスクは32-7=25、つまり/25となる。

\subsection{Step2:ネットワークアドレスを求める}
Officeの次、10.0.20.0/25 から始まる。

\subsubsection{暗算テクニック}
/25は第四オクテットで128ずつ増加。10.0.20.0, 10.0.20.128, ...
Labの次のネットワークアドレスは10.0.20.128。

\section{Linkの設計}
\subsection{Step1:サブネットマスクを求める}
Linkは2台のホストが必要。サブネットマスクは $2^n-2 \geq 2$ を満たす最小のnを求める。
\begin{align*}
    2^{2} - 2 &= 2 \\
    2^{1} - 2 &= 0
\end{align*}
よって、n=2が必要。サブネットマスクは32-2=30、つまり/30となる。

\subsection{Step2:ネットワークアドレスを求める}
Labの次、10.0.20.128/30 から始まる。

\subsubsection{暗算テクニック}
/30は第四オクテットで4ずつ増加。10.0.20.128, 10.0.20.132, ...
Linkの次のネットワークアドレスは10.0.20.132。
\begin{align*}
    2^{12} - 2 &= 4094 \\
    2^{11} - 2 &= 2046
\end{align*}
よって、n=12が必要。サブネットマスクは32-12=20、つまり/20となる。

\subsection{Step2:ネットワークアドレスを求める}
これは簡単
元のネットワークアドレスをそのまま使う。

\subsubsection{暗算テクニック}
次のネットワークアドレスを最初に計算すると早く求められる。

・考え方
次のネットワークはOffice。Factoryのサブネットは/20。
↓
/20ということは、第三オクテッドでネットワークが切られている。
↓
/20の第三オクテッドは\[2^4 = 16\]で、16ずつ増加する。
↓
もとのネットワーク(10.0.0.0/20)の第三オクテッドに16を足すと次のネットワークアドレスが求められる。
↓ つまり
次のネットワークアドレスは、10.0.16.0から始まる。このネットワークアドレスの一つ前が、Factoryのブロードキャストアドレスとなる。
このフローで考えれば、慣れれば暗算可能。

\end{document}
